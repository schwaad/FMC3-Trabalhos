\section*{3. (Máximo Divisor Comum). O programa a seguir afirma encontrar o máximo divisor comum mdc(a,b) de dois inteiros positivos a e b. Prove que a fórmula bem formada (fbf) está totalmente correta.}
$\{(a > 0) \land (b > 0)\}$
\begin{algorithmic}
	\State $x := a$
	\State $y := b$
	\While{$x \neq y$}
	\If{$x > y$}
	\State $x := x - y$
	\Else
	\State $y := y - x$
	\EndIf
	\EndWhile
	\State $max := x$
\end{algorithmic}
$\{max = \operatorname{mdc}(a,b)\}$
\vspace{0.5cm}

Provarei que a fbf está totalmente correta usando o **Invariante de Laço ($I$):**
\[ I \equiv (\operatorname{mdc}(x,y) = \operatorname{mdc}(a,b)) \land (x > 0) \land (y > 0) \]
e a **Função Variante ($t$):** $t = x + y$.

\vspace{0.5cm}

\subsection*{I. Inicialização (Antes do laço)}

Devemos provar que a pré-condição implica o invariante após as atribuições iniciais.

\begin{flalign*}
	 & 1. \quad \{(\operatorname{mdc}(a,b) = \operatorname{mdc}(a,b)) \land (a > 0) \land (b > 0)\} \ x:=a; y:=b \ \{I\}      & \text{AA (múltiplo)}    \\
	 & 2. \quad a > 0 \land b > 0                                                                                             & \text{Pré-condição (P)} \\
	 & 3. \quad \operatorname{mdc}(a,b) = \operatorname{mdc}(a,b)                                                             & \text{Reflexividade}    \\
	 & 4. \quad (a > 0 \land b > 0) \rightarrow ((\operatorname{mdc}(a,b) = \operatorname{mdc}(a,b)) \land a > 0 \land b > 0) & 2, 3, \text{Lógica}     \\
	 & 5. \quad \{a > 0 \land b > 0\} \ x:=a; y:=b \ \{I\}                                                                    & 1, 4, \text{Conseq}
\end{flalign*}

\vspace{0.5cm}

\subsection*{II. Manutenção (Durante a execução do laço)}

Devemos provar que $\{I \land B\} \text{ Corpo } \{I\}$. O corpo possui um condicional.
Seja $M = \operatorname{mdc}(a,b)$. O invariante é $mdc(x,y)=M \land x,y > 0$. A guarda é $x \neq y$.

\noindent \textbf{Caso A: Se $(x > y)$ é verdadeiro (Ramo 'If')}
\begin{flalign*}
	 & 6. \quad \{(\operatorname{mdc}(x-y, y) = M) \land (x-y > 0) \land (y > 0)\} \ x := x - y \ \{I\}       & \text{AA}                            \\
	 & 7. \quad I \land (x \neq y) \land (x > y)                                                              & \text{Hipótese no Ramo If}           \\
	 & 8. \quad \operatorname{mdc}(x,y) = M                                                                   & 7, \text{Simp de } I                 \\
	 & 9. \quad \operatorname{mdc}(x-y, y) = \operatorname{mdc}(x,y) = M                                      & \text{Propriedade do MDC (Euclides)} \\
	 & 10. \quad x > y \implies x - y > 0                                                                     & 7, \text{Aritmética}                 \\
	 & 11. \quad (I \land x > y) \rightarrow ((\operatorname{mdc}(x-y, y) = M) \land (x-y > 0) \land (y > 0)) & 9, 10, \text{Lógica}                 \\
	 & 12. \quad \{I \land x > y\} \ x := x - y \ \{I\}                                                       & 6, 11, \text{Conseq}
\end{flalign*}

\noindent \textbf{Caso B: Se $\neg(x > y)$ é verdadeiro (Ramo 'Else')}
Como $x \neq y$ (pela guarda) e não é $x > y$, então $y > x$. A prova é simétrica ao Caso A.
\begin{flalign*}
	 & 13. \quad \{I \land y > x\} \ y := y - x \ \{I\} & \text{Análogo aos passos 6-12}
\end{flalign*}

\noindent Portanto, o invariante se mantém independente do caminho tomado no \textit{If}.

\vspace{0.5cm}

\subsection*{III. Término (Correção Total)}

Para provar que o laço não é infinito, usamos a função variante $t = x + y$ com domínio nos Naturais.

\begin{flalign*}
	 & 14. \quad x > 0 \land y > 0 \implies x+y > 0                                               & \text{Do Invariante } I      \\
	 & 15. \quad \text{Se } x > y: \ t_{new} = (x-y) + y = x < x+y = t_{old}                      & \text{Decrescimento estrito} \\
	 & 16. \quad \text{Se } y > x: \ t_{new} = x + (y-x) = y < x+y = t_{old}                      & \text{Decrescimento estrito} \\
	 & 17. \quad \text{O laço termina pois } t \text{ decresce e é limitado inferiormente por 0.} & \text{Conclusão}
\end{flalign*}

\vspace{0.5cm}

\subsection*{IV. Finalização (Depois do laço)}

Quando o laço termina, a guarda é falsa ($x = y$) e o invariante $I$ ainda é verdadeiro.

\begin{flalign*}
	 & 18. \quad \{ max = \operatorname{mdc}(a,b) \} \ max := x \ \{ max = \operatorname{mdc}(a,b) \} & \text{AA (inválido - atribuição direta)} \\
	 & \quad \textit{Correção do passo lógico final para atribuição:}                                 &                                          \\
	 & 19. \quad \{ x = \operatorname{mdc}(a,b) \} \ max := x \ \{ max = \operatorname{mdc}(a,b) \}   & \text{AA}                                \\
	 & 20. \quad I \land \neg B                                                                       & \text{Estado pós-laço}                   \\
	 & 21. \quad (\operatorname{mdc}(x,y) = \operatorname{mdc}(a,b)) \land (x = y)                    & 20, \text{Subst}                         \\
	 & 22. \quad \operatorname{mdc}(x,x) = x                                                          & \text{Propriedade Aritmética}            \\
	 & 23. \quad x = \operatorname{mdc}(a,b)                                                          & 21, 22, \text{Transitividade}            \\
	 & 24. \quad (I \land \neg B) \rightarrow (x = \operatorname{mdc}(a,b))                           & 20\text{-}23, \text{PC}                  \\
	 & 25. \quad \{I \land \neg B\} \ max := x \ \{max = \operatorname{mdc}(a,b)\}                    & 19, 24, \text{Conseq}
\end{flalign*}

\textbf{Resolução de forma indutiva:}
Considere $W$ como o conjunto dos números naturais e a função $f(x,y) = x + y$. Usaremos $ < $.
Assuma que $mdc(a,b) = mdc(x,y)$ e que $x \neq y$.
Seja $s = (x,y)$. Então $f(s) = x + y$. Teremos duas possibilidades para $t$, de acordo com $x$ e $y$ na última iteração:

\begin{enumerate}
	\item Se $x < y$, então $t = (x,\, y - x)$ e f(t) = x + (y - x) = y.

	\item Se $x > y$, então $t = (x - y,\, y)$ e f(t) = (x - y) + y = x.
\end{enumerate}

Como $a > 0$ e $b > 0$, temos $x > 0$ e $y > 0$. Em ambas as situações acima,
temos $f(t) < f(s)$, pois $x < x + y$ e $y < x + y$.

Portanto, o laço termina.


O laço terminará quando x = y e, ao fim do laço, max := x. Ou seja, max := x - y ou max := a (caso x != y na primeira iteração).

O laço termina quando $(x=y)$. No momento da saída do enquanto, temos $(x=y>0)$ e $mdc(x,y)=mdc(x,x)=mdc(x-y, y)$. Sabemos que $mdc(a, b)=mdc(x-y, y)$.
O programa atribui então $(max := x)$,
portanto $(max=mdc(a,b))$. Assim (\{P\}\ S\ \{Q\}) é verdadeiro:
se a execução termina, a pós-condição $(mdc(a,b)=max)$ é satisfeita.
