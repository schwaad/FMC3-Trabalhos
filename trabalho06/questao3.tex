\section*{Questão 3}

Dê uma prova formal, nas duas notações (tabelas e árvores), para cada uma das tautologias a seguir usando a regra CP.

\noindent
(a) $(A \;\lor B \rightarrow C) \;\land \; (C \rightarrow D \;\land E) \rightarrow (A \rightarrow D)$

\subsubsection*{Tabela}
\begin{flalign*}
&1.\quad A \lor B \rightarrow C && \tag*{[Premissa]} &\\
&2.\quad C \rightarrow D \land E && \tag*{[Premissa]} &\\
&3.\quad\qquad A && \tag*{[Premissa (para $A\rightarrow D$)]} &\\
&4.\quad\qquad A \lor B && \tag*{[3, Ad]} &\\
&5.\quad\qquad C && \tag*{[1,4\; MP]} &\\
&6.\quad\qquad D \land E && \tag*{[2,5\; MP]} &\\
&7.\quad\qquad D && \tag*{[6, Simp]} &\\
&8.\quad A \rightarrow D && \tag*{[3--7, PC]} &\\
&9.\quad \text{QED} && \tag*{[1,2,8, PC]} &
\end{flalign*}

\subsubsection*{Árvore}

\[
\AxiomC{$A \vee B \to C$ (Suposição)}
\AxiomC{$A$ (Suposição)}                       
\UnaryInfC{$A \vee B$ (Adição)}    
\BinaryInfC{$C$ (MP)}              
\AxiomC{$C \to D \land E$}          
\BinaryInfC{$D \land E$ (MP)}      
\UnaryInfC{$D$ (Simplificação)}
\UnaryInfC{$A \to D$ (PC1)}
\UnaryInfC{$(A \vee B \to C) \land (C \to D \land E) \to (A \to D)$ (PC0)}
\DisplayProof
\]

\noindent
(b) $(A \rightarrow C) \rightarrow (A \rightarrow B \; \lor C)$

\subsubsection*{Tabela}
\begin{flalign*}
&1.\quad A \rightarrow C && \tag*{[Premissa]} &\\
&2.\quad\qquad A && \tag*{[Premissa (para $A \rightarrow B \lor C$)]} &\\
&3.\quad\qquad C && \tag*{[1,2 MP]} &\\
&4.\quad\qquad C \lor B && \tag*{[3, Ad]} &\\
&5.\quad\qquad B \lor C && \tag*{[Exercício de sala]} &\\
&6.\quad A \rightarrow (B \lor C) && \tag*{[2--5, PC]} &\\
&7.\quad \text{QED} && \tag*{[1,6, PC]} &
\end{flalign*}

\subsubsection*{Árvore}

\[
\AxiomC{$A \to C$}                 
\AxiomC{$A$ (Suposição)}                        
\BinaryInfC{$C$ (MP)}               
\AxiomC{$B$ (Suposição)}                        
\BinaryInfC{$B \vee C$ (Casos)}     
\UnaryInfC{$A \to (B \vee C)$ (PC1)}  
\UnaryInfC{$(A \to C) \to (A \to (B \vee C))$ (PC0)} 
\DisplayProof
\]


\noindent
(c) $(A \rightarrow B) \rightarrow (C \lor A \rightarrow C \lor B)$

\subsubsection*{Tabela}
\begin{flalign*}
&1.\quad A \rightarrow B && \tag*{[Premissa]} &\\
&2. \quad \qquad C \lor A && \tag*{[Premissa (para $C \lor A \rightarrow C \lor B$]} \\
&3. \quad \qquad \qquad A && \tag*{Premissa (para $A \rightarrow B \lor C$)} \\
&4. \quad \qquad \qquad B && \tag*{[1,3. MP]} \\
&5. \quad \qquad  B \lor C && \tag*{[4. Ad]} \\
&6. \quad \qquad \qquad C \\
&7. \quad \qquad \qquad C \lor B && \tag*{[6. Ad]} \\
&8. \quad \qquad  B \lor C && \tag*{[Exercício de sala]} \\
&9. \quad  C \lor A \rightarrow B \lor C && \tag*{[2-8. PC]} \\
&10. \quad QED && \tag*{[1,9. PC]} 
\end{flalign*}

\subsubsection*{Árvore}
% Caso 1: C
\begin{prooftree}
    \AxiomC{$C$ (Suposição)}
    \UnaryInfC{$C \vee B$ (Adição)}
\UnaryInfC{$C \vee A \to C \vee B$ (PC1)}
\end{prooftree}

% Caso 2: A
\begin{prooftree}
    \AxiomC{$A$ (Suposição)}
    \AxiomC{$A \to B$}
    \BinaryInfC{$B$ (MP)}
    \UnaryInfC{$C \vee B$ (Adição)}
\UnaryInfC{$C \vee A \to C \vee B$ (PC1)}
\end{prooftree}

% Conclusão
\begin{prooftree}
    \AxiomC{$A \to B$}
    \AxiomC{$C \vee A$}
    \BinaryInfC{$C \vee B$ (Casos)}
    \UnaryInfC{$(A \to B) \to ((C \vee A) \to (C \vee B))$ (PC0)}
\end{prooftree}
