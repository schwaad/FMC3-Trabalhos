\section*{Questão 1}

Dadas as fórmulas:

\begin{enumerate}[label=\roman*)]
	\item $Q \land \lnot P \rightarrow P$
	\item $(P \lor Q) \land R$
\end{enumerate}

\noindent
(a) Use equivalências para transformá-las em FNC.

\subsubsection*{i.}

\begin{align*}
	(Q \land \lnot P) \rightarrow P
	 & \Leftrightarrow \lnot (Q \land \lnot P) \lor P      \\[6pt]
	 & \Leftrightarrow (\lnot Q \lor \lnot \lnot P) \lor P \\[6pt]
	 & \Leftrightarrow (\lnot Q \lor P) \lor P             \\[6pt]
	 & \Leftrightarrow (\lnot Q \lor P)
\end{align*}

\subsubsection*{ii.}

\begin{align*}
	(P \lor Q) \land R
	 & \Leftrightarrow (R \land P) \lor (R \land Q)
\end{align*}


\newpage
\noindent
(b) Use equivalências para transformá-las em uma FNC.

\subsubsection*{i.}

\begin{align*}
	(Q \land \lnot P) \rightarrow P                         \\
	 & \Leftrightarrow \lnot (Q \land \lnot P) \lor P       \\
	 & \Leftrightarrow \lnot Q \land \lnot(\lnot P) \land P \\
	 & \Leftrightarrow \lnot Q \land P \land P              \\
	 & \Leftrightarrow \lnot Q \land P
\end{align*}

\subsubsection*{ii.}
A fbf $(P \lor Q) \land R$ já é uma FNC, pois é a conjunção de uma disjunção de literais e um literal.

\newpage

\noindent
(c) Transforme as fbfs em uma FND completa, se possível

\subsubsection*{i.}
Podemos fazer a transformação por meio do método da tabela verdade, utilizando a equivalência demonstrada anteriormente:


$ \ $


\begin{tabular}{|c|c|c|c|}
	\hline
	$Q$ & $P$ & $\lnot Q$ & $\lnot Q \lor P$ \\
	\hline
	$V$ & $V$ & $F$       & $V$              \\
	\hline
	$V$ & $F$ & $F$       & $F$              \\
	\hline
	$F$ & $V$ & $V$       & $V$              \\
	\hline
	$F$ & $F$ & $V$       & $V$              \\
	\hline
\end{tabular}


$ \ $


Portanto, a FND completa é:


$(Q \land P) \lor (\lnot Q \land P) \lor (\lnot Q \land \lnot P)$
\subsubsection*{ii.}
Podemos fazer a transformação por meio do método da tabela verdade, utilizando a equivalência demonstrada anteriormente:


$ \ $


\begin{tabular}{|c|c|c|c|}
	\hline
	$P$ & $Q$ & $R$ & $(R \land P) \lor (R \land Q)$ \\
	\hline
	$V$ & $V$ & $V$ & $V$                            \\
	\hline
	$F$ & $V$ & $V$ & $V$                            \\
	\hline
	$V$ & $F$ & $V$ & $V$                            \\
	\hline
	$V$ & $V$ & $F$ & $F$                            \\
	\hline
	$F$ & $F$ & $V$ & $F$                            \\
	\hline
	$F$ & $V$ & $F$ & $F$                            \\
	\hline
	$V$ & $F$ & $F$ & $F$                            \\
	\hline
	$F$ & $F$ & $F$ & $F$                            \\
	\hline
\end{tabular}


$ \ $


Portanto, a FND completa é:


$(P \land Q \land R) \lor (\lnot P \land Q \land R) \lor (P \land \lnot Q \land R)$

\newpage
\noindent
(d) Transforme as fbfs em uma FNC completa, se possível

\subsubsection*{i.}
A fbf $\lnot Q \land P$ já é uma FNC completa, pois é a conjunção de dois literais. Segue a prova por tabela verdade:


$ \ $


\begin{tabular}{|c|c|c|c|}
	\hline
	$Q$ & $P$ & $\lnot Q$ & $\lnot Q \land P$ \\
	\hline
	$V$ & $V$ & $F$       & $F$               \\
	\hline
	$F$ & $V$ & $V$       & $V$               \\
	\hline
	$V$ & $F$ & $F$       & $F$               \\
	\hline
	$F$ & $F$ & $V$       & $F$               \\
	\hline
\end{tabular}
\subsubsection*{ii.}
Podemos fazer a transformação por meio do método da tabela verdade, utilizando a equivalência demonstrada anteriormente:


$ \ $


\begin{tabular}{|c|c|c|c|}
	\hline
	$P$ & $Q$ & $R$ & $(P \lor Q) \land R$ \\
	\hline
	$V$ & $V$ & $V$ & $V$                  \\
	\hline
	$F$ & $V$ & $V$ & $V$                  \\
	\hline
	$V$ & $F$ & $V$ & $V$                  \\
	\hline
	$V$ & $V$ & $F$ & $F$                  \\
	\hline
	$F$ & $F$ & $V$ & $F$                  \\
	\hline
	$F$ & $V$ & $F$ & $F$                  \\
	\hline
	$V$ & $F$ & $F$ & $F$                  \\
	\hline
	$F$ & $F$ & $F$ & $F$                  \\
	\hline
\end{tabular}


$ \ $


Portanto, a FNC completa é:

$(P \lor Q \lor R) \land (\lnot P \lor Q \lor R) \land (P \lor \lnot Q \lor R) \land (P \lor Q \lor \lnot R) \land (\lnot P \lor \lnot Q \lor R)$

