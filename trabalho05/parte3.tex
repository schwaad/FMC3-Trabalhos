\textbf{5. Mostre que todo conjunto finito A é bem fundado sob qualquer reação transitiva e irreflexiva sobre ele}

\begin{proof}
	Seja $A$ finito e $R$ transitiva e irreflexiva sobre $A$:

	Suponha por absurdo que A não é bem fundado sob R.

	Seja S uma sequência infinita decrescente $S = (s_1,s_2,...,s_i)$

	\[
		\exists (s_i): s_{i+1}Rs_i
		\;\implies\; \exists j>k: s_j=s_k \quad (\text{$A$ finito, princípio da casa dos pombos})
	\]
	\[
		\implies s_jRs_k \quad (\text{pela transitividade na cadeia $s_jRs_{j-1}\cdots Rs_k$})
	\]
	\[
		\implies s_jRs_j \quad (\text{pois $s_j=s_k$})
	\]
	\[
		\implies \text{contradição} \quad (\text{irreflexividade de $R$}).
	\]
	Portanto, $A$ é bem fundado.
\end{proof}

\bigskip

\textbf{6. Seja A qualquer conjunto finito. Mostre que seu conjunto de potências $\mathscr{P}(A)$ está bem fundado na relação de inclusão apropriada de conjunto.}

\begin{proof}
	Seja $S=(A_i)$ cadeia descendente em $\mathscr{P}(A)$:
	\[
		A_{i+1}\subset A_i
		\;\implies\; |A_{i+1}|<|A_i| \quad (\text{Def. Cardinalidade})
	\]
	\[
		\implies |A| > \cdots > |A_2| > |A_1| \ge 0 \quad (\text{Def. P)}
	\]
	\[
		\implies \text{sequência finita} \quad (\text{Começa em A e termina em } \emptyset).
	\]
	Logo, $\mathscr{P}(A)$ é bem fundado sob $\subset$.
\end{proof}