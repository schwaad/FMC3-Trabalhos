\subsection*{4. Prove o Teorema da Dedução para o cálculo de predicado.}
O Teorema da Dedução pode ser definido da seguinte forma:
$$\text{Se } \Gamma, A \vdash B \text{, então } \Gamma \vdash A \rightarrow B$$
Para provar, façamos uma prova por indução sob o comprimento $n$ do cálculo lógico. 
\\
\textbf{1. Caso Base ($n=1$)}
\\
No caso base, $B$ é a primeira linha da prova de $\Gamma, A \vdash B$. Há duas possibilidades: \\
1. $B \in \Gamma$ \\
2. $B$ é a premissa $A$. \\
Façamos uma demonstração por casos: \\
\\
\textbf{Caso 1: $B \in \Gamma$ (ou $B$ é um Axioma)}
\begin{flalign*}
&1. \Gamma \vdash B \tag*{P}\\
&2. \Gamma \vdash B \rightarrow (A \rightarrow B) \tag*{Axioma} \\
&3. \Gamma \vdash A \rightarrow B \tag*{1, 2, MP} \\
&QED
\end{flalign*}

\textbf{Caso 2: $B$ é a premissa $A$}
\begin{flalign*}
&1. \Gamma \vdash A \rightarrow A \tag*{Tautologia} \\
&2. \Gamma \vdash A \rightarrow B \tag*{B = A} \\
&QED 
\end{flalign*}
Com isso, tem-se que a base é válida.
\\
\textbf{2. Passo Indutivo} \\
Hipótese indutiva: O teorema é válido para todo $k < n$ \\
Novamente, temos duas possibilidades sobre como $B$ foi derivado: \\
1. Modus Ponens ($\text{MP}$). \\
2. Generalização Universal ($\text{GU}$). \\
Novamente, façamos uma demonstração por casos: \\

\textbf{Modus Ponens} \\
$B$ é derivada de $C$ e $C \rightarrow B$ por $\text{MP}$.
\begin{flalign*}
&1. \Gamma \vdash A \rightarrow C \tag*{HI} \\
&2. \Gamma \vdash A \rightarrow (C \rightarrow B) \tag*{HI} \\
&3. \Gamma \vdash (A \rightarrow (C \rightarrow B)) \rightarrow ((A \rightarrow C) \rightarrow (A \rightarrow B)) \tag*{Axioma} \\
&4. \Gamma \vdash (A \rightarrow C) \rightarrow (A \rightarrow B) \tag*{2, 3, MP} \\
&5. \Gamma \vdash A \rightarrow B \tag*{1, 4, MP} \\
&QED
\end{flalign*}

\textbf{Generalização Universal} \\
$B = \forall x C$ é derivada de $C$ por $\text{GU}$ (assumindo que $x$ não é livre em $A$ ou em $\Gamma$). A dedução de $C$ a partir de $\Gamma, A$ tem comprimento $< n$.
\begin{flalign*}
&1. \Gamma \vdash A \rightarrow C \tag*{HI} \\
&2. \Gamma \vdash \forall x (A \rightarrow C) \tag*{1, GU} \\
&3. \Gamma \vdash \forall x (A \rightarrow C) \rightarrow (A \rightarrow \forall x C) \tag*{Axioma} \\
&4. \Gamma \vdash A \rightarrow \forall x C \tag*{2, 3, MP} \\
&5. \Gamma \vdash A \rightarrow B \tag*{4, Substituição ($B = \forall x C$)} \\
&QED 
\end{flalign*}

Assim, prova-se o teorema da dedução.