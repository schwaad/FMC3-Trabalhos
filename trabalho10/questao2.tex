\subsection*{2. Prove que a regra de prova do cálculo proposicional Silogismo Hipotético pode ser usada no
	cálculo de predicado.}

Assumimos que as premissas $A \rightarrow B$ e $B \rightarrow C$ são fbfs válidas.


Devemos mostrar que $A \rightarrow C$ também é uma fbf válida


Suponha que temos uma interpretação $I$ arbitrária sobre um domínio $D$


Assim, segue-se que $A \rightarrow B$ é verdadeiro com respeito à interpretação $I$ e $B \rightarrow C$ é verdadeiro com respeito à interpretação $I$


Portanto, usando a fbf de predicado como instância de fbf proposicional, temos as seguintes premissas verdadeiras:


Valor verdade de A $\rightarrow$ Valor verdade de B


Valor verdade de B $\rightarrow$ Valor verdade de C


Como sabemos que o Silogismo Hipotético é uma tautologia, podemos concluir que:


Valor verdade de A $\rightarrow$ Valor verdade de C


E portanto a fórmula $A \rightarrow C$ é verdadeira com respeito à interpretação $I$ dada sobre $D$


Portanto, o Silogismo Hipotético pode ser usado no cálculo de predicado.
