\documentclass[12pt]{article}
\usepackage{amsmath,amssymb,amsthm}
\usepackage[utf8]{inputenc}
\usepackage[portuguese]{babel}
\usepackage{algorithm}
\usepackage{algpseudocode}

\title{FMC III - Trabalho 4}
\author{Alexandre Ribeiro, José Ivo e Marina Medeiros}
\date{Setembro 2025}

\begin{document}

\maketitle

\textbf{1. Mostre que uma partição de um conjunto é pelo menos tão fina quanto outra sse a relação de equivalência associada com a primeira é uma sub-relação da relação de equivalência associada com a última.}

\begin{proof}
    Seja $A$ um conjunto e sejam $\{P_i\}_{i \in I}$ e $\{Q_j\}_{j \in J}$ duas partições de $A$.  
    Sejam $R_P$ e $R_Q$ as relações de equivalência associadas a $\{P_i\}_{i \in I}$ e $\{Q_j\}_{j \in J}$, respectivamente.

    \medskip
    Queremos mostrar que:
    \[
        \{P_i\}_{i \in I} \text{ é pelo menos tão fina quanto } \{Q_j\}_{j \in J}
        \;\;\Longleftrightarrow\;\;
        R_P \subseteq R_Q.
    \]

    \medskip
    ($\Rightarrow$) Suponha que $\{P_i\}_{i \in I}$ seja pelo menos tão fina quanto $\{Q_j\}_{j \in J}$. Isso significa que $\forall i \in I, \exists j \in J; P_i \subseteq Q_j$. Com isso, teremos que:
    
    \begin{align*}
        \exists i \in I; a, b \in P_i
        &\implies [(a, b) \in R_P] \land [\exists j \in J; P_i \subseteq Q_j] \\
        &\implies [(a, b) \in R_P] \land [a, b \in Q_j] \\
        &\implies [(a, b) \in R_P] \land [(a, b) \in R_Q] \\
        &\implies R_P \subseteq R_Q \\
    \end{align*}

    \medskip
    ($\Leftarrow$) Suponha que $R_P \subseteq R_Q$.  
    
    \begin{align*}
        (a,b) \in R_P
        &\implies [\exists i \in I, a, b \in P_i] \land [(a, b) \in R_Q] \\
        &\implies [\exists i \in I, a, b \in P_i] \land [\exists j \in J, a, b \in Q_j] \\
        &\implies \forall i \in I, \exists j \in J; P_i \subseteq Q_j
    \end{align*}




    \medskip
    Concluímos que:
    \[
        \{P_i\}_{i \in I} \text{ é pelo menos tão fina quanto } \{Q_j\}_{j \in J}
        \;\;\Longleftrightarrow\;\;
        R_P \subseteq R_Q.
    \]
\end{proof}

\textbf{2.  A intersecção dos fechos transitivos de duas relações é sempre igual ao fecho transitivo da
sua intersecção? Se for verdadeiro, prove, se for falso, de um exemplo onde não vale.}

\begin{proof}
Falso. Tome como contra-exemplo as relações:


$R = \{(1,2), (2,3)\}$


$S = \{(1,3)\}$


Em um conjunto $A = {1,2,3}$


$t(R) \cap t(S):$
\begin{align*}
t(R) \cap t(S) = 
\\ t(\{(1,2),(2,3)\}) \cap t(\{(1,3)\}) = 
&\text{(Def. de R e S)}\\ \{(1,2),(2,3),(1,3)\} \cap \{(1,3)\} = 
&\text{(Def. Fecho Transitivo)} \\ \{(1,3)\} 
\end{align*}

$T(R \cap S):$
\begin{align*}
t(R \cap S) = 
\\ t(\{(1,2),(2,3)\} \cap \{(1,3)\}) =
&\text{(Def. de R e S)}\\ t(\emptyset) = 
&\text{(Def. } \cap)\\ \emptyset
&\text{ (Def. Fecho transitivo)}\end{align*}



Logo, a intersecção dos fechos transitivos de duas relações não é sempre igual ao fecho transitivo de sua intersecção.
\end{proof}

\textbf{3. Prove que se $R$ é uma relação binária sobre A, então $tsr(R)$ é a menor relação de equivalência que contém $R$.}

\begin{proof}
    Seja R uma relação. Vamos demonstrar que tsr(R) contém R.

    \begin{align}
        tsr(R) 
        &\Longleftrightarrow t(s(r(R))) \\
        &\Longleftrightarrow t((s(R \; \cup R^0))) \tag{Corolário: r(R) = R $\cup$ R$^0$ } \\
        &\Longleftrightarrow t((R \; \cup R^0) \; \cup (R \;\cup R^0)^{-1}) \tag{Corolário: s(R) = R $\cup$ R$^{-1}$ } \\
        &\Longleftrightarrow t(R \; \cup R^0 \; \cup R^{-1} \; \cup (R^0)^{-1} \tag{Corolário: (R $\cup$ S)$^{-1}$ = R$^{-1} \; \cup$ S$^{-1}$} \\
        &\Longleftrightarrow t(R \; \cup R^0 \; \cup R^{-1} \cup R^0) \tag{(R$^0)^{-1} = R^0$} \\
        &\Longleftrightarrow t(R \; \cup R^0 \; \cup R^{-1})   \tag{Idempotência} \\
        &= \bigcup_{n=1}^{\infty} (R \cup R^{-1} \cup R^0)^{n} \tag{Definição de fecho transitivo}
    \end{align}

Assim. Sabemos que R $\subset R$ pois todo conjunto contém a si mesmo.

    \begin{align}
        R \subset R
        &\implies R \subset R \cup R^{-1} \cup R^0  \tag{Idempotência} \\
        &\implies R \subset (R \cup R^{-1} \cup R^0)^1 \\
        &\implies R \subset tsr(R)
    \end{align}

Dessa forma, temos que tsr(R) contém R. \\
Além disso, pela própria definição de fecho, sabemos que tsr(R) é uma relação de equivalência, visto que aplica as propriedades de reflexividade, simetria e transitividade. 
\\
\\
Agora, demonstraremos que tsr(R) é a menor relação de equivalência que contém R.\\
\\
Suponha uma relação de equivalência S qualquer tal que R $\subset$ S
\begin{align}
    R \subset S
    & \implies r(R) \subset S \tag{S já é reflexiva } \\
    & \implies sr(R) \subset S \tag{S já é simétrica} \\
    & \implies tsr(R) \subset S \tag{S já é transitiva}
\end{align}
Concluindo, tsr(R) sempre estará contida em uma relação de equivalência que contém R, ou seja, é a menor possível. 

\end{proof}

\end{document}
