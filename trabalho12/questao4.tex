\section*{4. Calcule a composição $\theta \sigma$ de cada um dos pares de substituições a seguir}
\[
\theta = \{x/f(z), y/a\}, \sigma = \{z/b\}.
\]
\text{1. Aplique $\sigma$ aos termos (denominadores) de $\theta$. Resulta em}
\[
\{x/f(z)\sigma ,y/a \sigma\} =\{x/f(b),y/a\}
\]
\text{2. Exclua quaisquer ligações no formato $x_i/x_i$ do conjunto do Passo 1. Resulta em}
\[
\{x/f(b),y/a\}
\]
\text{3. Exclua qualquer $y_i/s_i$ de $\sigma$ se $y_i$ for uma variável em $dom(\theta) = \{x, y\}$.}
\\
O conjunto $\sigma = \{z/b\}$. A variável $z$ não pertence ao domínio $\{x, y\}$ de $\theta$.
\[
\{z/b\}
\]
\text{4. $\theta \sigma$ é a união dos conjuntos construídos dos Passos 2 e 3}
\[
\theta \sigma = \{x/f(b), y/a, z/b\}
\]
\\
\textbf{Verificação}
\\
Para verificar, usaremos o átomo $p(x,y,z)$, pois $x, y$ e $z$ são as variáveis distintas que ocorrem no domínio e contradomínio de $\theta$ e $\sigma$.
\\
\text{Primeiro cálculo:}
\[
(p(x,y,z)\theta)\sigma = p(f(z),a,z)\sigma = p(f(b),a,b)
\]
\text{Segundo cálculo:}
\[
p(x,y,z)(\theta \sigma) = p(f(b),a,b)
\]
Assim. como os resultados são os mesmos, isso confirma que a composição está correta.